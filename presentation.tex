% LaTeX Beamer template, made with linguistics in mind
% 
% * Intended to be compiled with XeLaTeX: <http://www.xelatex.org/>
% * Uses the free Libertine font: <http://www.linuxlibertine.org/>
%   (defines character U+203A in top bar)
% * This file calls packages: booktabs, fontspec, gb4e, natbib, newtxmath,
%                             tabularx, tikz, tikz-qtree
% 
% Feel free to use or modify this for any purpose, but please keep the following
% attribution and license in the source code. Attribution is not required in the
% compiled PDF.


% The MIT License (MIT)
% 
% Copyright (c) 2014 Alexander Klapheke
% 
% Permission is hereby granted, free of charge, to any person obtaining a copy
% of this software and associated documentation files (the "Software"), to deal
% in the Software without restriction, including without limitation the rights
% to use, copy, modify, merge, publish, distribute, sublicense, and/or sell
% copies of the Software, and to permit persons to whom the Software is
% furnished to do so, subject to the following conditions:
% 
% The above copyright notice and this permission notice shall be included in all
% copies or substantial portions of the Software.
% 
% THE SOFTWARE IS PROVIDED "AS IS", WITHOUT WARRANTY OF ANY KIND, EXPRESS OR
% IMPLIED, INCLUDING BUT NOT LIMITED TO THE WARRANTIES OF MERCHANTABILITY,
% FITNESS FOR A PARTICULAR PURPOSE AND NONINFRINGEMENT. IN NO EVENT SHALL THE
% AUTHORS OR COPYRIGHT HOLDERS BE LIABLE FOR ANY CLAIM, DAMAGES OR OTHER
% LIABILITY, WHETHER IN AN ACTION OF CONTRACT, TORT OR OTHERWISE, ARISING FROM,
% OUT OF OR IN CONNECTION WITH THE SOFTWARE OR THE USE OR OTHER DEALINGS IN THE
% SOFTWARE.


\PassOptionsToPackage{leqno,fleqn}{amsmath} % to prevent option clash
\documentclass[xetex,serif,xcolor=x11names,compress]{beamer}
\usetheme{clean}

% Sets title, author, etc.
\title[Title]{Title of My Presentation}
\subtitle{Subtitle}
\author[Klapheke]{Alexander~Klapheke}
\institute[Harvard]{Harvard~University}
\date{November 14, 2012}

% font
\usepackage{fontspec}
\usepackage[libertine]{newtxmath}
\setromanfont[Mapping=tex-text,Numbers=Lining]{Linux Libertine O}
\setmonofont[Scale=MatchLowercase]{Inconsolata Plus 0}

% Strongly discourage hyphenation
\hyphenpenalty=5000
\tolerance=1000

% citations
\usepackage[round,sort&compress]{natbib}

% nice-looking tables
\RequirePackage{booktabs}
% tables with relative fixed-width. use code below to allow wrapping
\RequirePackage{tabularx}
\newcolumntype{C}{>{\centering\arraybackslash}X}
\newcolumntype{L}{>{\raggedright\arraybackslash}X}

% highlighting
\newcommand\hla[1]{\textcolor{CustomRed}{#1}}  % primary color
\newcommand\hlb[1]{\textcolor{CustomBlue}{#1}} % secondary color

%%%%%%%%%%%%%%%%%%%%%%%%%%%%%%%%%%%%%%%
% Linguistics-specific things follow: %
%%%%%%%%%%%%%%%%%%%%%%%%%%%%%%%%%%%%%%%

% glosses
\usepackage{gb4e}
\primebars % use primes instead of 
\let\eachwordone=\it % italicize original in glosses
\resetcounteronoverlays{exx} % don't increment example numbers on \pause

% for arrows in glosses
\newcommand{\tikzmark}[1]{\tikz[overlay,remember picture] \node (#1) {};}

% for trees
\usepackage{tikz}
\usepackage{tikz-qtree}

% Uncomment to start in presentation mode automatically
\hypersetup{%
	colorlinks=false,
	% pdfpagemode=FullScreen,
}

\begin{document}

\section{}
\begin{frame}
	\titlepage
\end{frame}

\section{Section}
\subsection{Outline}
\begin{frame}{Outline}
	\vfill

	Lorem ipsum dolor sit amet, \hlb{consectetur adipisicing elit}, sed do eiusmod tempor incididunt ut labore et dolore magna aliqua. \citep{aspects}

	\vfill

	\begin{itemize}
		\item Duis aute irure dolor in reprehenderit in voluptate
		\item $\left\llbracket\textsc{Imperf}\right\rrbracket=\mathrm{\lambda}P_{\langle l,st\rangle}\,.\,\mathrm{\lambda}t\,.\,\mathrm{\lambda}w\,.\,\exists e\left[t\subseteq \tau(e)\land P(e,w)=1\right]$
		\item Velit esse cillum dolore eu fugiat nulla pariatur
	\end{itemize}

	\vfill

	John \tikz[baseline=(a.base),remember picture]\node(a){\hla{can play}}; the guitar,
	and Mary \tikz[baseline=(b.base),remember picture]\node(b){\hla{Δ}}; the violin.
	\begin{tikzpicture}[overlay,remember picture,>=latex]
		\draw[semithick,->,out=315,in=225,looseness=0.5,color=CustomRed] (a.south) to (b.south);
	\end{tikzpicture}

	\vfill
\end{frame}

\subsection{Gloss with IPA}
\begin{frame}{Gloss with IPA}
	\begin{exe}
		\ex{\label{ex:french} {\glll Sont des mots qui vont très bien ensemble. \\
				sɔ̃ de mo ki vɔ̃ tʁɛ bjɛ̃ ɑ̃sɑ̃bl \\
				be.\textsc{3p} \textsc{part} word-\textsc{pl} \textsc{rel.nom} go.\textsc{3pl} very well together \\
			\trans `These are words that go together well.'}
		}
	\end{exe}
\end{frame}

\subsection{Table}
\begin{frame}{Table}
	\newcommand\tablep{\hlb{\Large\textbf{+}}}
	\newcommand\tablem{\hla{\Large\textbf{−}}}
	% \newcommand\tablep{\hlb{\large ✔}}
	% \newcommand\tablem{\hla{\large ✘}}
	\begin{tabularx}{\linewidth}{lCCCC} \toprule
		Verb type      & Stative & Durative & Telic   & Subinterval \\ \midrule
		State          & \tablep & \tablep  & \tablem & \tablep     \\
		Activity       & \tablem & \tablep  & \tablem & \tablep     \\
		Accomplishment & \tablem & \tablep  & \tablep & \tablem     \\
		Achievement    & \tablem & \tablem  & \tablep & \tablem     \\
		Semelfactive   & \tablem & \tablem  & \tablem & \tablem     \\ \bottomrule
	\end{tabularx}
\end{frame}

\subsection{Tree}
\begin{frame}{Tree}
	\begin{center}
		\begin{tikzpicture}
			\tikzset{every tree node/.style={align=center,anchor=north}}
			\Tree [.TP
				[.DP \edge[roof]; \node(s){\hlb{Socrates}}; ]
				[.T$'$
					[.T \textit{pres} ]
					[.VP
						[.\node(t){\hla{$t$}}; ]
						[.V$'$
							[.V \hlb{is} ]
							[.AdjP \edge[roof]; \hlb{mortal} ]
						]
					]
				]
			]
			\hla{\draw[semithick,->,out=225,in=270,looseness=1.25,shorten >=1pt,shorten <=1pt] (t.south west) to (s.south);}
		\end{tikzpicture}
	\end{center}
\end{frame}

\subsection{Graph}
\begin{frame}{Graph}
	\begin{center}
		\begin{tikzpicture}[y=0.80pt, x=0.8pt,yscale=-1, inner sep=0pt, outer sep=0pt]
			\begin{scope}[shift={(-76.26843,-728.64282)}]
				\path[fill=CustomLightBlue,line width=3.200pt,rounded corners=0.0000cm] (165.9949,346.3301) rectangle (188.9356,359.2472);
				\path[fill=CustomLightBlue,line width=3.200pt,rounded corners=0.0000cm] (191.4461,329.2797) rectangle (214.3868,359.2472);
				\path[fill=CustomLightBlue,line width=3.200pt,rounded corners=0.0000cm] (216.8975,299.8289) rectangle (239.8381,359.2472);
				\path[fill=CustomLightBlue,line width=3.200pt,rounded corners=0.0000cm] (242.3487,260.0444) rectangle (265.2893,359.2472);
				\path[fill=CustomLightBlue,line width=3.200pt,rounded corners=0.0000cm] (267.7999,212.5097) rectangle (290.7405,359.2472);
				\path[fill=CustomLightBlue,line width=3.200pt,rounded corners=0.0000cm] (293.2511,172.2086) rectangle (316.1918,359.2472);
				\path[fill=CustomLightBlue,line width=3.200pt,rounded corners=0.0000cm] (318.7023,140.1744) rectangle (341.6429,359.2472);
				\path[fill=CustomLightBlue,line width=3.200pt,rounded corners=0.0000cm] (344.1535,120.5405) rectangle (367.0942,359.2471);
				\path[fill=CustomLightBlue,line width=3.200pt,rounded corners=0.0000cm] (369.6047,110.7236) rectangle (392.5454,359.2472);
				\path[draw=CustomRed,miter limit=4.00,line width=2.294pt] (144.7901,358.0976) ..
					controls (169.0610,354.9835) and (193.4679,343.9341) .. (213.7952,320.4327) ..
					controls (246.3223,283.4134) and (269.4273,228.4684) .. (297.7569,183.4606) ..
					controls (316.2834,153.2288) and (338.4470,127.8906) .. (363.7537,117.9230) ..
					controls (373.2925,113.7858) and (383.0380,111.4973) .. (392.8358,110.0519);
				\path[draw=CustomPurple,line join=miter,line cap=butt,miter limit=4.00,line width=2.294pt] (143.2401,109.0748) -- (143.2401,361.7143);
				\path[draw=CustomPurple,line join=miter,line cap=butt,miter limit=4.00,line width=1.434pt] (144.4740,110.0784) -- (156.8743,110.0784);
				\path[draw=CustomPurple,line join=miter,line cap=butt,miter limit=4.00,line width=1.434pt] (144.4740,135.5297) -- (156.8743,135.5297);
				\path[draw=CustomPurple,line join=miter,line cap=butt,miter limit=4.00,line width=1.434pt] (144.4740,160.9809) -- (156.8743,160.9809);
				\path[draw=CustomPurple,line join=miter,line cap=butt,miter limit=4.00,line width=1.434pt] (144.4740,186.4321) -- (156.8743,186.4321);
				\path[draw=CustomPurple,line join=miter,line cap=butt,miter limit=4.00,line width=1.434pt] (144.4740,211.8833) -- (156.8743,211.8833);
				\path[draw=CustomPurple,line join=miter,line cap=butt,miter limit=4.00,line width=1.434pt] (144.4740,237.3345) -- (156.8743,237.3345);
				\path[draw=CustomPurple,line join=miter,line cap=butt,miter limit=4.00,line width=1.434pt] (144.4740,262.7858) -- (156.8743,262.7858);
				\path[draw=CustomPurple,line join=miter,line cap=butt,miter limit=4.00,line width=1.434pt] (144.4740,288.2370) -- (156.8743,288.2370);
				\path[draw=CustomPurple,line join=miter,line cap=butt,miter limit=4.00,line width=1.434pt] (144.4740,313.6882) -- (156.8743,313.6882);
				\path[draw=CustomPurple,line join=miter,line cap=butt,miter limit=4.00,line width=1.434pt] (144.4740,339.1394) -- (156.8743,339.1394);
				\path[draw=CustomPurple,line join=miter,line cap=butt,miter limit=4.00,line width=2.294pt] (394.8463,360.6809) -- (142.2067,360.6809);
				\path[draw=CustomPurple,line join=miter,line cap=butt,miter limit=4.00,line width=1.434pt] (393.8426,359.4471) -- (393.8426,347.0467);
				\path[draw=CustomPurple,line join=miter,line cap=butt,miter limit=4.00,line width=1.434pt] (368.3914,359.4471) -- (368.3914,347.0467);
				\path[draw=CustomPurple,line join=miter,line cap=butt,miter limit=4.00,line width=1.434pt] (342.9402,359.4471) -- (342.9402,347.0467);
				\path[draw=CustomPurple,line join=miter,line cap=butt,miter limit=4.00,line width=1.434pt] (317.4890,359.4471) -- (317.4890,347.0467);
				\path[draw=CustomPurple,line join=miter,line cap=butt,miter limit=4.00,line width=1.434pt] (292.0377,359.4471) -- (292.0377,347.0467);
				\path[draw=CustomPurple,line join=miter,line cap=butt,miter limit=4.00,line width=1.434pt] (266.5865,359.4471) -- (266.5865,347.0467);
				\path[draw=CustomPurple,line join=miter,line cap=butt,miter limit=4.00,line width=1.434pt] (241.1353,359.4471) -- (241.1353,347.0467);
				\path[draw=CustomPurple,line join=miter,line cap=butt,miter limit=4.00,line width=1.434pt] (215.6841,359.4471) -- (215.6841,347.0467);
				\path[draw=CustomPurple,line join=miter,line cap=butt,miter limit=4.00,line width=1.434pt] (190.2328,359.4471) -- (190.2328,347.0467);
				\path[draw=CustomPurple,line join=miter,line cap=butt,miter limit=4.00,line width=1.434pt] (164.7816,359.4471) -- (164.7816,347.0467);
			\end{scope}
		\end{tikzpicture}
	\end{center}
\end{frame}

\section{References}
\begin{frame}{References}
	% Can replace following with: \bibliography{presentation}
	\begin{thebibliography}{1}
		\bibitem[{Chomsky(1965)}]{aspects}
			Chomsky, N. 1965.
			\newblock \textit{Aspects of the Theory of Syntax}.
			\newblock Cambridge, MA: MIT Press
	\end{thebibliography}
\end{frame}

\section{}
\begin{frame}
	\begin{center}
		{\Huge Thank you!}
	\end{center}
\end{frame}

\end{document}
